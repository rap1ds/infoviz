\documentclass[a4paper]{article}

% Encoodaus, joka sopii suomenkielellä (esim. ä ja ö)
\usepackage[utf8]{inputenc}
\usepackage[T1]{fontenc}

% Suomenkielinen tavutus
\usepackage[finnish]{babel}

% Viitteet
\usepackage{natbib}

\usepackage{alltt}

% Otsikkojen päätteetön fontti
% \usepackage{sectsty}
% \allsectionsfont{\sffamily\large}

% Viitteiden merkit
\bibpunct{(}{)}{;}{a}{,}{,}

\usepackage{titlesec}
\titleformat{\subsection}{\normalsize\bfseries}{}{0em}{}{}
\titlespacing{\subsection}{0pt}{*1.5}{*1.0}

\begin{document}

\title{\small Yrityksen rahoitus \\ \huge Tenttivastaukset}
\date{25.2.2012}
\author{Mikko Koski \\ mikko.koski@aalto.fi \\ 66467F}
\maketitle

\normalsize
\setlength{\parindent}{0cm}

\section{Tentti 2011 - 05}

\subsection{Sensory symbols}

\textbf{Sensory symbols} are symbols or aspects of visualizations that can be understood without learning. A meaning of the sensory symbols tends to be stable across individuals, cultures and time. Example: A cave drawing of a hunt.

As opposite a \textbf{arbitrary symbols} must be learned, for example written word "dog".

\subsection{Instructional bias}

\subsection{CIE system of color standards}

\subsection{Chromaticity diagram}

\subsection{Pre-attentive features ++}



Discuss pre-attentive features in design and use of glyphs

\subsection{Conjunction search of two pre-attentive attributes}

\subsection{PCA ++++}

What is the principal component?

Describe principal component analysis (PCA) in a suffient detail to understand the working ideas and principles

Descrube the properties of the visualization produced by PCA, and demonstrate the difference in resulting visualizations using a small toy example

\subsection{Curvilinear Component Analysis, CCA +}

True or false? CCA does not preserve small distances

\subsection{Cost function for CCA +}

Formula?

\subsection{Euclidean distance (between points)}

\subsection{Unity (CCA)}

\subsection{Depth Cue theory +++++}
Explain

List and explain depth cues

\subsection{Dimensionality reduction +++++}

a) Describe, define and discuss the measures that can be used to evaluate the "goodnes" of the projection.

What to take into account when choosing a method to use

\subsection{Precision (in dimensionality reduction)}

\subsection{Chart Junk}

\subsection{Gibson's Affordance Theory ++}

\subsection{Self-Organizing Maps (for dimensionality reduction)}

\subsection{Saccadic Movements}

\subsection{Pre-processing steps}

\subsection{Post-processing steps}

\subsection{Quality criteria}

\subsection{Visual attention theory ++}

Ways that allow information to pop-up

\subsection{Active from the low-level point of view}

\section{Tentti 2010 - 03}

\subsection{Bad/good recall +}

\subsection{Bad/good precision +}

\subsection{MDS Multidimensional Scaling ++++}

Describe multidimenstional scaling (MDS) in a suffient detail to understand the working ideas and principles

Descrube the properties of the visualization produced by MDS, and demonstrate the difference in resulting visualizations using a small toy example

True or false? Multidimensional scaling preserves small distances

True of false? For MDS, the stress is always increasing with the projection dimension

True of false? MDS requires the coordinates of the original data to project (distances are not enough)

\subsection{Laplacian Eigenmap LE}

ToF? Projecting to one dimension using LE is enough to preserce the proximity coded in L

ToF? Projecting to two dimension using LE is enough to preserce the proximity coded in L

ToF? Projecting to three dimension using LE is enough to preserce the proximity coded in L

ToF? Laplacian Eigenmap does not work in this case

\subsection{Lie-factor}

\subsection{Data-ink}

\subsection{Gestalt laws +++++}

What are gestalt laws and how can they be useful?

List and briefly describe them.

\subsection{Geons +}

\subsection{Pre-attentive processing}

\subsection{History and theory of data graphics according to Tufte ++++++}

History of data graphics and lessons learned (before the time of the computers)

\section{Tentti 2009 - 05}

\subsection{Luminance and brightness +}

\subsection{CIE XYZ, CIELab and CIELuv +++}

\subsection{Image based object recognition ++}

Theory and implications in information visualization

\subsection{Structure based object recognition ++}

Theory and implications in information visualization

\subsection{Semiotics of graphics ++}

\subsection{Glyph design +++}

Design a glyph, taking the properties of human perception into account

Design goals and choices from the viewpoint of human perception

Design goals and choices from the viewpoint of data graphics

Design goals and choices from the viewpoint of resulting visualization



\section{Tentti 2009 - 03}

\subsection{Gamut +}

\subsection{Superacuity +}

\subsection{Focus and context problem ++}

Methods to solve it

\subsection{Parallel coordinates}

\subsection{Color scales}

Important issues in designing color scales. Discuss pros and cons of the rainbow color scale as an example

\section{Tentti 2008 - 05}

\subsection{Moire effects}

\subsection{Effective view navigation}

\section{Tentti 2008 - 03}

\subsection{Chernoff face}

\subsection{Cost-knowledge characteristic function +}

\subsection{ISOMAP, isometric mapping od data manifolds}

\subsection{How the human eye resembles and differs from a camera}

\subsection{Perception of visual patterns and objects}

\section{Tentti 2007 - 05}

\subsection{Crispening}

\subsection{How to lay out a node-link diagram: principles and algorithms}

\subsection{Aesthetic principles}

In graph drawing, give some examples

\end{document}