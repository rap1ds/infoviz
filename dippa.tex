\documentclass[a4paper]{article}

% Encoodaus, joka sopii suomenkielellä (esim. ä ja ö)
\usepackage[T1]{fontenc}
\usepackage[utf8]{inputenc}

% Suomenkielinen tavutus
\usepackage[finnish]{babel}

% Viitteet
\usepackage{natbib}

\usepackage{alltt}

% Otsikkojen päätteetön fontti
% \usepackage{sectsty}
% \allsectionsfont{\sffamily\large}

% Viitteiden merkit
\bibpunct{(}{)}{;}{a}{,}{,}

\usepackage{titlesec}
\titleformat{\subsection}{\normalsize\bfseries}{}{0em}{}{}
\titlespacing{\subsection}{0pt}{*1.5}{*1.0}

\begin{document}

\title{\small Yrityksen rahoitus \\ \huge Tenttivastaukset}
\date{25.2.2012}
\author{Mikko Koski \\ mikko.koski@aalto.fi \\ 66467F}
\maketitle

\normalsize
\setlength{\parindent}{0cm}

\section{Tentti 2011 - 05}

\subsection{Sensory symbols}

\textbf{Sensory symbols} are symbols or aspects of visualizations that can be understood without learning. A meaning of the sensory symbols tends to be stable across individuals, cultures and time. Example: A cave drawing of a hunt.

Processing of sensory symbols is fast and hard-wired. They are resist instructional bias.

As opposite a \textbf{arbitrary symbols} must be learned, for example written word "dog".

\subsection{Instructional bias}

Because of instructional bias, we perceive a different line length because of the environment where the line is.

\subsection{Arbitrary symbols}

Hard to learn, easy to extend. Culture specific

\subsection{CIE system of color standards}

\subsection{Chromaticity diagram}

\subsection{Pre-attentive features ++}



Discuss pre-attentive features in design and use of glyphs

\subsection{Conjunction search of two pre-attentive attributes}

\subsection{PCA ++++}

What is the principal component?

Describe principal component analysis (PCA) in a suffient detail to understand the working ideas and principles

Descrube the properties of the visualization produced by PCA, and demonstrate the difference in resulting visualizations using a small toy example

\subsection{Curvilinear Component Analysis, CCA +}

True or false? CCA does not preserve small distances

\subsection{Cost function for CCA +}

Formula?

\subsection{Euclidean distance (between points)}

\subsection{Unity (CCA)}

\subsection{Depth Cue theory +++++}

Depth cue is a source of information about 3D space. Visual system processes these cues and provides a perception of space. 

Three categories (based on how they can be reproduced:
\begin{itemize}
  \item Monocular static - static picture
  \item Monocular dynamic - moving picture
  \item Binocular - requires two eyes
\end{itemize}

Monocular static (Pictorial)
\begin{itemize}
  \item Linear perspective - Geometry, ray from each point in the env
  \item Texture gradient - Uniformly textured surface projected onto the picture plane.
  \item Size gradient - More distant objects become smaller
  \item Occlusion - Object overlapping another appears closer to the observer
  \item Depth of focus - Eye focus on one object, the (more distant) background becomes blurry
  \item Cast shadows - Bouncing ball
  \item Shape-from-shading - Use of artificial light source
  \item Depth-from-eye accommondation (nonpictorial) 
\end{itemize}

Monocular dynamic (Moving picture)
\begin{itemize}
  \item Structure-from-motion (kinectic depth, motion parallax)
  Motion parallax: Looking out of a train window. Objects closer to us move faster than objects closer to the horizon. 
  Kinectic depth: Bent 3D wire projected onto a screen. Result is a 2D line, but if rotated the 3D shape of the wire come apparent.
\end{itemize}

Binocular
\begin{itemize}
  \item Eye convergence
  \item Stereoscopic depth
\end{itemize}

No unified cognitive theory for space perception

Experimental evidence that several depth cues are involved and weighted differently for different perception tasks.

Moon illusion: The moon appears larger when it is close to the horizon than when it is overhead. There are many theories as to why, but no single accepted explanation.


List and explain at least 5 depth cues

Occlusion - If one object overlaps or occludes another, it appears closer to the observer. A strong, but binary, detph cue.

Relative Size - Bigger objects appear to be closer to the observer.

Linear Height - Objects closer to the horizon line appear further away.

Linear Perspective - Related to both relative size and linear height. The geometry of linear perspective is obtained by sending a ray from each point in the environment through a picture plane to a single fixed point.

Aerial Perspective (=distance fog) - As the distance between an object and a viewer increases, the contrast between the object and its background decreases, and the contrast of any markings or details within the object also decreases. The colours of the object also become less saturated and shift towards the background color (usually blue).

\subsection{Dimensionality reduction +++++}

a) Describe, define and discuss the measures that can be used to evaluate the "goodnes" of the projection.

What to take into account when choosing a method to use

\subsection{Precision (in dimensionality reduction)}

\subsection{Chart Junk}

\subsection{Gibson's Affordance Theory ++}

Affordance (ilmeisyys): Perceivalbe possibilities for action. Gibson said the "we perceive in order to operate". We perceive possibilities for action. 

Theory is attractive for visualization, because the goal of visualization is decision making. 

According to Gibson, affordances are physical properties of the environment that we directly percive.

1) Visualization is very abstract and indirect
2) No clear physical affordance: GUI button is very arbitrary
3) Rejection of visual mechanism

Influental theory?

\subsection{Self-Organizing Maps (for dimensionality reduction)}

\subsection{Saccadic Movements}

\subsection{Pre-processing steps}

\subsection{Post-processing steps}

\subsection{Quality criteria}

\subsection{Visual attention theory ++}

Ways that allow information to pop-up



\subsection{Active from the low-level point of view}

\section{Tentti 2010 - 03}

\subsection{Bad/good recall +}

\subsection{Bad/good precision +}

\subsection{MDS Multidimensional Scaling ++++}

Describe multidimenstional scaling (MDS) in a suffient detail to understand the working ideas and principles

Descrube the properties of the visualization produced by MDS, and demonstrate the difference in resulting visualizations using a small toy example

True or false? Multidimensional scaling preserves small distances

True of false? For MDS, the stress is always increasing with the projection dimension

True of false? MDS requires the coordinates of the original data to project (distances are not enough)

\subsection{Laplacian Eigenmap LE}

ToF? Projecting to one dimension using LE is enough to preserce the proximity coded in L

ToF? Projecting to two dimension using LE is enough to preserce the proximity coded in L

ToF? Projecting to three dimension using LE is enough to preserce the proximity coded in L

ToF? Laplacian Eigenmap does not work in this case

\subsection{Lie-factor}

\subsection{Data-ink}

\subsection{Gestalt laws +++++}

What are gestalt laws and how can they be useful?

List and briefly describe them.

\subsection{Pre-attentive processing}

\subsection{History and theory of data graphics according to Tufte ++++++}

History of data graphics and lessons learned (before the time of the computers)

\section{Tentti 2009 - 05}

\subsection{Luminance and brightness +}

\subsection{CIE XYZ, CIELab and CIELuv +++}

\subsection{Geons +}

(Hummel and Biederman) Geon theory suggests that image recognition process has different processing stages (layers). Visual information is decomposed first into edges then axes, oriented blobs and vertices. At the next layer 3D primitives (cones, cylinders...) called \textbf{geons} are identified.

Object recognition is achieved by finally analysing the interconnections of the geons.

\subsection{Image based object recognition ++}

Theory and implications in information visualization

People can recognice object from images remarkable well. If we show a pictures very rapidly (10 pic per sec), people are able to tell if there were a dog in one of the picture (Rapid serial visual presentation, RSVP).

Related phenomen: attentional blink, brains are processing still the first dog

Recognition ++ vs. recall --.

Objects are best recognized if they are in the same direction as they were initially seen. Also, opposite (silhouette) is well recognized

Priming effect: recognition is easier if prior exposure (altistuminen) to relevant information is given. For example, recognition of a horse is easier if picture of a cow is shown before horse.

Palmer et al (1981): Many objects have a canonical view from which they are most easily identified. We recognize objects by matching the visual information with "key views"

Implications in visualization:

Images easily recognized: Use icons

Priming effect: helping people search for particular patterns in data.

RSVP can be utilized for searching images from database (quickly show 10 images per second. This is a lot faster than scanning thumbnail grid)



\subsection{Structure based object recognition ++}

Theory and implications in information visualization

We recognize object not only based on the image but based on the structure, should as silhouettes of the object. Two identical objects in a different angle can be easily recognized as the same object despite the very different visual image.

As a consequnce a line drawing of a hand may be easier to recognize than a picture of a hand

Implications in visualizations:

(See the geon theory)

If geons (cylinders, cones etc.) are indeed perceptual primitives, it will make sense to construct diagrams using these geons (geon diagram). 

Geons can be used to visualize relationships in a natural way: because of gravity, above is different from below. Object inside another object: representation of a part-of relationship. One object touching another: representation of dependency. Thick bar between objects: strong relationship. A thinner, transparent bar between objects: weak relationship.

\subsection{Semiotics of graphics ++}

\subsection{Glyph design +++}

Design a glyph, taking the properties of human perception into account

Design goals and choices from the viewpoint of human perception

Design goals and choices from the viewpoint of data graphics

Design goals and choices from the viewpoint of resulting visualization



\section{Tentti 2009 - 03}

\subsection{Gamut +}

\subsection{Superacuity +}

\subsection{Focus and context problem ++}

Methods to solve it

\subsection{Parallel coordinates}

\subsection{Color scales}

Important issues in designing color scales. Discuss pros and cons of the rainbow color scale as an example

\section{Tentti 2008 - 05}

\subsection{Moire effects}

\subsection{Effective view navigation}

\section{Tentti 2008 - 03}

\subsection{Chernoff face}

\subsection{Cost-knowledge characteristic function +}

\subsection{ISOMAP, isometric mapping od data manifolds}

\subsection{How the human eye resembles and differs from a camera}

\subsection{Perception of visual patterns and objects}

\section{Tentti 2007 - 05}

\subsection{Crispening}

\subsection{How to lay out a node-link diagram: principles and algorithms}

\subsection{Aesthetic principles}

In graph drawing, give some examples

\end{document}